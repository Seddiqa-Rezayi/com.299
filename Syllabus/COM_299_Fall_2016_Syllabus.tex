\documentclass[12pt,a4paper,oneside]{article}

\usepackage[margin=3cm]{geometry}

\usepackage{hyperref}
\hypersetup{
    pdftitle={COM 299, Game Development},%
    pdfauthor={Toksaitov Dmitrii Alexandrovich},%
    pdfsubject={Syllabus},%
    pdfkeywords={COM;}{299;}{syllabus;}{game;}{development},%
    colorlinks,%
    linkcolor=black,%
    citecolor=black,%
    filecolor=black,%
    urlcolor=black
}

\newcommand{\R}[1]{\uppercase\expandafter{\romannumeral #1\relax}}

\begin{document}

    \title{COM 299, Game Development}
    \author{
        American University of Central Asia\\
        Software Engineering Department
    }
    \date{}
    \maketitle

    \section{Course Information}

        \begin{description}
            \item[Course ID]\hfill\\
                COM 299, 3956
            \item[Course Repository]\hfill\\
                \url{https://github.com/auca/com.299}
            \item[Place]\hfill\\
                AUCA, laboratory G30
            \item[Time]\hfill\\
                Wednesday 10:50\\
                Friday 10:50
        \end{description}

    \section{Prerequisites}

        COM 117, Programming \R{2}. Object-oriented Design

    \section{Contact Information}

        \begin{description}
            \item[Instructor]\hfill\\
                Toksaitov Dmitrii Alexandrovich\\
                \href{mailto:toksaitov_d@auca.kg}{toksaitov\_d@auca.kg}
            \item[Office]\hfill\\
                AUCA, room 315
            \item[Office Hours]\hfill\\
                Monday 12:45--14:45\\
                Tuesday 10:50--12:45, 14:00--16:00\\
                Wednesday 12:45--14:45\\
                Friday 14:00--16:00
        \end{description}

    \section{Course Overview}

        The course introduces students to a topic of game development. It covers
        theory and practice of video game production. It delves into of fields
        of computer graphics, computational physics, artificial intelligence,
        and game-play design. During the course students will get an opportunity
        to build two market-ready games for desktop, web, or mobile platforms.
        They will not only learn how to create their own lightweight graphics,
        physics and game-play engines, but also how to use third-party solutions
        such as Unity or Unreal Engine.

    \section{Quizzes}

        Students will get four quizzes throughout the course on topics discussed
        during classes.

    \section{Presentation}

        Students will have to make one presentation about a game of their
        choice.  The presentation should be focused on the internals of the
        game, development or production process, tools or techniques.

    \section{Course Projects}

        Students will have to finish three course projects.\\

        In the first project students will modify an implementation of a classic
        old game created during practice classes into a different similar game
        of their choice. Both games will be created with the help of the Unity
        engine. All scripts will be written in C\#.\\

        In the second project students will implement several subsystems of a
        toy 3-D real-time OpenGL 2.0 rendering engine written in C++. Students
        will also write a number of programs for the GPU in GLSL. At the end,
        students will port their Unity game from the first project to the newly
        created engine.\\

        The final project is similar to the first one, except students are free
        to use any other engine (e.g., Unreal 4). The complexity of the game
        will be much higher and students will get an opportunity to try out as
        much systems of the selected game engine as possible.

    \section{Reading}

        3D Math Primer for Graphics and Game Development, Second Edition by
        Fletcher Done and Ian Parberry (ISBN: 978-1-4398-6981-9)

    \section{Grading}

        \begin{itemize}
            \item Quizzes (20\%)
            \item Presentation (20\%)
            \item Course projects (60\%)
        \end{itemize}

        \begin{itemize} \itemsep-10pt \parskip0pt \parsep0pt
            \item[--] 90\%--100\%: A\\
            \item[--] 80\%--89\%: A-\\
            \item[--] 70\%--79\%: B+\\
            \item[--] 65\%--69\%: B\\
            \item[--] 60\%--64\%: B-\\
            \item[--] 56\%--59\%: C+\\
            \item[--] 53\%--55\%: C\\
            \item[--] 50\%--52\%: C-\\
            \item[--] 46\%--49\%: D+\\
            \item[--] 43\%--45\%: D\\
            \item[--] 40\%--42\%: D-\\
            \item[--] Less than 39\%: F
        \end{itemize}

    \section{Rules}

        Students are required to follow the rules of conduct of the Software
        Engineering Department and American University of Central Asia.

        Team work is NOT encouraged. The same blocks of code or similar
        structural pieces in separate works will be considered as academic
        dishonesty and all parties will get zero for the task.

\end{document}

